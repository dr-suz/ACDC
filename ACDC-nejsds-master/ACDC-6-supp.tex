
\section{Notation}
Let $N(\bx;\bmu,\Sigma)$ be the normal density at $\bx$
with mean $\bmu$ and variance $\Sigma$, and $\ftil_{n}(\bs\mid\btheta)=N\{\bs;\bs(\btheta),A(\btheta)/a_{n}^{2}\}$, the asymptotic distribution of the summary statistic.
We define $a_{n,\veps}=a_{n}$ if $\lim_{n\rightarrow\infty}a_{n}\veps_{n}<\infty$
and $a_{n,\veps}=\veps_{n}^{-1}$ otherwise, and $\ensuremath{c_{\veps}=\lim_{n\rightarrow\infty}a_{n}\veps_{n}}$,
%%ST TO DO: Our Thm 2 assumes \veps_n = o(a_n^{-1})
both of which summarize how $\veps_{n}$ decreases relative to the
converging rate, $a_{n}$, of $S_{n}$ in Condition 2 below. Define
the standardized random variables $W_{n}(S_{n})=a_{n}A(\theta)^{-1/2}\{S_{n}-s(\theta)\}$
and $W_{\rm obs}=a_{n}A(\theta)^{-1/2}\{s_{\rm obs}-s(\theta_0)\}$ and $\beta_{0}=I(\theta_0)^{-1}Ds(\theta_0)^{T}A(\theta_0)^{-1}$ according to Condition 2 below. 

Let $f_{W_{n}}(w\mid\theta)$ and
$\ftil_{W_{n}}(w\mid\theta)$ be the density for $W_{n}(S_{n})$ when
$S_{n}\sim f_{n}(\cdot\mid\theta)$ and $\ftil_n(\cdot\mid\theta)$ respectively.
Let $B_{\delta}=\{\theta\mid\|\theta-\theta_{0}\|\leq\delta\}$ for
$\delta>0$. Define the initial density truncated in $B_{\delta}$,
i.e. $r_{n}(\theta)\mathbb{I}_{\theta\in B_{\delta}}/\int_{B_{\delta}}r_{n}(\theta)\,d\theta$,
by $r_{\delta}(\theta)$. Let $t(\theta)=a_{n,\veps}(\theta-\theta_{0})$
and $v(s)=\veps_{n}^{-1}(s - s_{\rm obs})$. For any $A\in\mathscr{B}^{p}$
where $\mathscr{B}^{p}$ is the Borel sigma-field on $\mathbb{R}^{p}$,
let $t(A)$ be the set $\{\phi:\phi=t(\theta)\text{ for some }\theta\in A\}$.
For a non-negative function $h(x)$, integrable in $\mathbb{R}^{l}$,
denote the normalized function $h(x)/\int_{\mathbb{R}^{l}}h(x)\,dx$
by $h(x)^{({\rm norm})}$. For a function $h(x)$, denote its gradient
by $D_{x}h(x)$, and for simplicity, omit $\theta$ from $D_{\theta}$. For a sequence $x_n$, we use the notation $x_n = \Theta(a_n)$ to mean that there exist some constants $m$ and $M$ such that $0<m<\mid x_n/a_n \mid<M<\infty$.


\section{Conditions}

%=\begin{condition} \label{sum_conv}
{\it Condition 2.} There exists a sequence $a_{n}$, satisfying $a_{n}\rightarrow\infty$
	as $n\rightarrow\infty$, a $d$-dimensional vector $s(\btheta)$
	and a $d\times d$ matrix $A(\btheta)$, such that for $S_{n}\sim f_{n}(\cdot\mid\theta)$
	and all $\btheta\in\mathcal{P}_{0}$, 
	\[
	a_{n}\{\bS_{n}-s(\btheta)\}\rightarrow N\{0,A(\btheta)\},\mbox{ as \ensuremath{n\rightarrow\infty}},
	\]
	in distribution. We also assume that $\bs_{\rm obs}\rightarrow s(\btheta_{0})$
	in probability. Furthermore, assume
	
	 (i) $s(\btheta)$
	and $A(\btheta)\in C^{1}(\mathcal{P}_{0})$, and $A(\btheta)$ is
	positive definite for any $\btheta$; 
	
	 (ii) for any $\delta>0$ there
	exists a $\delta'>0$ such that $\|s(\btheta)-s(\btheta_{0})\|>\delta'$
	for all $\btheta$ 
	
	satisfying $\|\btheta-\btheta_{0}\|>\delta$; and
	
	 (iii) $I(\theta)\stackrel{\hbox{\tiny def}} = \left\{ \frac{\partial}{\partial\theta}s(\theta)\right\} ^{T}A(\theta)^{-1}\left\{ \frac{\partial}{\partial\theta}s(\theta)\right\} $
	has full rank at $\btheta=\btheta_{0}$.
%\end{condition}


\noindent {\it Condition 3.} There exists some $\delta_0 > 0$ such that $\mathcal{P}_{\delta_0} = \{\theta: \| \theta-\theta_0\|  < \delta_0  \} \subset \mathcal{P},$ 
For all $\theta \in \mathcal{P}_{0},$ $r_{n}(\theta) \in C^2(\mathcal{P}_{0})$ and $r_{n}(\theta_0)>0$.

	
\noindent {\it Condition 4.} There exists a sequence $\{\tau_{n}\}$ %and $\delta>0$, 
such that $\tau_{n}=o(a_n)$ and $\sup_{\btheta\in {\cal P}_{0}}\tau_{n}^{-p}r_{n}(\btheta)=O_{p}(1)$.


\noindent {\it Condition 5.} There exists constants $m$, $M$ such that $0 < m <\mid \tau_{n}^{-p}r_{n}(\btheta_{0})\mid < M < \infty$.


\noindent {\it Condition 6.} It holds that $\sup_{\btheta\in\mathbb{R}^{p}}\tau_{n}^{-1} D\{\tau_{n}^{-p}r_{n}(\btheta)\}=O_{p}(1)$.

\begin{condition} \label{kernel_prop}
	The kernel satisfies 
	
	 (i) $\int vK_{\veps}(v)dv=0$; 
	
	 (ii)$\prod_{k=1}^{l}v_{i_{k}}K_{\veps}(v)dv<\infty$
	for any coordinates $(v_{i_{1}},\dots,v_{i_{l}})$ of $v$ and $l\leq p+6$;
	
	 (iii)$K_{\veps}(v)\propto K_{\veps}(\|v\|_{\Lambda}^{2})$ where $\|v\|_{\Lambda}^{2}=v^{T}\Lambda v$
	and $\Lambda$ is a positive-definite matrix, and $K(v)$ is a decreasing
	function of $\|v\|_{\Lambda}$; (iv) $K_{\veps}(v)=O(\exp\{-c_{1}\|v\|^{\alpha_{1}}\})$
	for some $\alpha_{1}>0$ and $c_{1}>0$ as $\|v\|\rightarrow\infty$. 
\end{condition}

\begin{condition} \label{sum_approx}
	There exists $\alpha_{n}$ satisfying $\alpha_{n}/a_{n}^{2/5}\rightarrow\infty$
	and a density $r_{max}(w)$ satisfying Condition \ref{kernel_prop}(ii)--(iii) where $K_{\veps}(v)$
	is replaced with $r_{max}(w)$, such that $\sup_{\theta\in B_{\delta}}\alpha_{n}\mid f_{W_{n}}(w\mid\theta)-\ftil_{W_{n}}(w\mid\theta)\mid\leq c_{3}r_{max}(w)$
	for some positive constant $c_{3}$. 
\end{condition}

\begin{condition} \label{sum_approx_tail}
	The following statements hold: 
	
	(i) $r_{max}(w)$ satisfies
	Condition \ref{kernel_prop}(iv); and 
	
	(ii) $\sup_{\theta\in B_{\delta}^{C}}\ftil_{W_{n}}(w\mid\theta)=O(e^{-c_{2}\|w\|^{\alpha_{2}}})$
	as $\|w\|\rightarrow\infty$ for some positive constants $c_{2}$
	and $\alpha_{2}$, and $A(\theta)$ is bounded in ${\cal P}$. 
\end{condition}

\begin{condition} \label{cond:likelihood_moments}
	The first two moments, $\int_{\mathbb{R}^{d}}s\ftil_{n}(s\mid\theta)ds$
	and $\int_{\mathbb{R}^d}s^{T}s\ftil_{n}(s\mid\theta)ds$, exist. 
\end{condition}


%-----------------------------------------------------------------------------------------------------------
\section{Proof for Theorem 2}

\noindent Let $\tilde{Q}(\theta\in A\mid s)=\int_{A}r_{\delta}(\theta)\ftil_{n}(s\mid\theta)\,d\theta/\int_{\mathbb{R}^{p}}r_{\delta}(\theta)\ftil_{n}(s\mid\theta)\,d\theta$. %\ST{should this function also include the kernel? }

\begin{lemma}\label{Alemma1} Assume Condition \ref{par_true}--\ref{sum_approx}. If $\veps_{n}=O(a_{n}^{-1})$, for any fixed $\nu\in\mathbb{R}^{d}$
	and small enough $\delta$, 
	\begin{eqnarray*}
	&&\hspace{-2cm}\sup_{A\in\mathfrak{B}^{p}}\big|\tilde{Q}\{a_{n}(\theta-\theta_{0})\in A\mid s_{\rm obs}+\veps_{n}\nu\}- \\
	&&\hspace{-1cm}\int_{A}N[t;\beta_{0}\{A(\theta_{0})^{1/2}W_{\rm obs}+c_{\veps}\nu\},I(\theta_{0})^{-1}]dt\big|\rightarrow0,
	\end{eqnarray*}
	in probability as $n\rightarrow\infty$, where $\beta_{0}=I(\theta_{0})^{-1}Ds(\theta_{0})^{T}A(\theta_{0})^{-1}$.
\end{lemma}

\noindent {\it Proof of Lemma \ref{Alemma1}:} 
	This result generalizes Lemma A1 in ~\cite{Li2017}. With Lemma A1 from ~\cite{Li2017}, it is sufficient to show that 
	\[
	\sup_{A\in\mathfrak{B}^{p}}\mid\tilde{Q}\{t(\theta)\in A\mid s_{\rm obs}+\veps_{n}\nu\}-\tilde{\Pi}\{t(\theta)\in A\mid s_{\rm obs}+\veps_{n}\nu\}\mid
	\]
	is $o_{P}(1)$ where $\tilde{\Pi}$ denotes the posterior distribution with prior $\pi_{\delta}(\theta)$ and likelihood $\tilde{f}_n(s \mid \theta)$. 
	%$\tilde{Q}$ using $r_{n}(\theta)$ rather than a prior $\pi(\theta)$ with a density satisfying Condition \ref{par_true}. 
With the transformation $t=t(\theta)$
	and $v=v(s)$, the left hand side of the above equation can be written
	as 
	\begin{eqnarray}
	&&\hspace{-0.7cm}\sup_{A\in\mathfrak{B}^{p}}\mid\frac{\int_{A}r_{\delta}(\theta_0+a_{n}^{-1}t)\ftil_{n}(s_{{\rm obs}}+\veps_{n}\nu\mid\theta_0+a_{n}^{-1}t)dt}{\int_{\mathbb{R}^{p}}r_{\delta}(\theta_0+a_{n}^{-1}t)\ftil_{n}(s_{{\rm obs}}+\veps_{n}\nu\mid\theta_0+a_{n}^{-1}t)dt}-\hfill  \label{eq1} \\
	&&\hspace{1cm}\frac{\int_{A}\pi_{\delta}(\theta_0+a_{n}^{-1}t)\ftil_{n}(s_{{\rm obs}}+\veps_{n}\nu\mid\theta_0+a_{n}^{-1}t)dt}{\int_{\mathbb{R}^{p}}\pi_{\delta}(\theta_0+a_{n}^{-1}t)\ftil_{n}(s_{{\rm obs}}+\veps_{n}\nu\mid\theta_0+a_{n}^{-1}t)dt}\mid. \nonumber 
	\end{eqnarray}
	For a function $\tau:\mathbb{R}^{p}\rightarrow\mathbb{R},$ define
	the following auxiliary functions,
	\begin{eqnarray*}
		&&\phi_{1}\{\tau(\theta);n\}  = \\ 
		&&\frac{\int_{t(B_{\delta})}|\tau(\theta_0+a_{n}^{-1}t)-\tau(\theta)|\ftil_{n}(s_{{\rm obs}}+\veps_{n}\nu\mid\theta_0+a_{n}^{-1}t)\,dt}{\int_{t(B_{\delta})}\tau(\theta_0+a_{n}^{-1}t)\ftil_{n}(s_{{\rm obs}}+\veps_{n}\nu\mid\theta_0+a_{n}^{-1}t)\,dt},\\
		&&\phi_{2}\{\tau(\theta);n\}  =  \\
		&&\frac{\tau(\theta)\int_{t(B_{\delta})}\ftil_{n}(s_{{\rm obs}}+\veps_{n}\nu\mid\theta_0+a_{n}^{-1}t)dt}{\int_{t(B_{\delta})}\tau(\theta_0+a_{n}^{-1}t)\ftil_{n}(s_{{\rm obs}}+\veps_{n}\nu\mid\theta_{0}+a_{n}^{-1}t)dt}.
	\end{eqnarray*}
	Then by adding and subtracting $\phi_{2}\{\tau_{n}^{-p}r_{\delta}(\theta);n\}\phi_{2}\{\pi(\theta);n\}$
	in the absolute sign of \eqref{eq1}, \eqref{eq1} can be bounded
	by 
	\begin{eqnarray*}
		&&\phi_{1}\{\tau_{n}^{-p}r_{\delta}(\theta);n\}+\phi_{1}\{\pi(\theta);n\}\phi_{2}\{\tau_{n}^{-p}r_{\delta}(\theta);n\} \\
		&&\quad +\phi_{1}\{\tau_{n}^{-p}r_{\delta}(\theta);n\}\phi_{2}\{\pi(\theta);n\}+\phi_{1}\{\pi(\theta);n\}.
	\end{eqnarray*}
	Consider a class of function $\tau(\theta)$ satisfying the following
	conditions: 
	
	\noindent (1) There exists a series $\{k_{n}\}$, such that $\sup_{\theta\in\mathcal{P}_{0}}\|k_{n}^{-1}D\tau(\theta)\|<\infty$
	and $k_{n}=o(a_{n});$ 
	
    \noindent (2)	$\tau(\theta_{0})>0$ and $\tau(\theta)\in C^{1}(B_{\delta}).$ 
	
	
	By Conditions \ref{par_true}--\ref{initial_gradient}, $\tau_{n}^{-p}r_{\delta}(\theta)$
	and $\pi_{\delta}(\theta)$ belong to the above class. Then if $\phi_{1}\{\tau(\theta);n\}$
	is $o_{p}(1)$ and $\phi_{2}\{\tau(\theta);n\}$ is $O_{p}(1)$, \eqref{eq1}
	is $o_{p}(1)$ and the lemma holds. 
	
	First, from the properties of $\tau(\theta)$ mentioned above, there exists an open set $\omega\subset B_{\delta}$
	such that $\inf_{\theta\in\omega}\tau(\theta)>c_{1}$, for a constant
	$c_{1}>0$. Then for $\phi_{2}\{\tau(\theta);n\}$, it is bounded
	by 
	\[
	\frac{\tau(\theta)}{c_{1}\int_{t(\omega)}\ftil_{n}(s_{{\rm obs}}+\veps_{n}\nu\mid\theta_{0}+a_{n}^{-1}t)^{(norm)}dt},
	\]
	where $h(x)^{(norm)}$ represents the normalized version of $h(x)$.
	From equation (7) in the supplementary material of \cite{Li2016},
	$\ftil_{n}(s_{{\rm obs}}+\veps_{n}\nu\mid\theta_0+a_{n}^{-1}t)$
	can be written in the following form, 
	\begin{eqnarray}
	&&\hspace{-1cm}a_{n}^{d}\ftil_{n}(s_{{\rm obs}}+\veps_{n}\nu\mid\theta_0+a_{n}^{-1}t)= \notag \\
	&&\hspace{-0.5cm}\frac{1}{\|B_{n}(t)\|^{1/2}}N[C_{n}(t)\{A_{n}(t)t-b_{n}\nu-c_{2}\};\theta_0,I_{d}],\label{eq2}
	%%%%%check out the math here, why the (1/2) in the norm? 
	\end{eqnarray}
	where $A_{n}(t)$ is a series of $d\times p$ matrix functions, $\{B_{n}(t)\}$
	and $\{C_{n}(t)\}$ are a series of $d\times d$ matrix functions,
	$b_{n}$ converges to a non-negative constant and $c_{2}$ is a constant,
	and the minimum of absolute eigenvalues of $A_{n}(t)$ and eigenvalues
	of $B_{n}(t)$ and $C_{n}(t)$ are all bounded and away from $0$.
	Then for fixed $\nu$, by continuous mapping, \eqref{eq2} is away
	from zero with probability one. Therefore $\phi_{2}\{\tau(\theta);n\}=O_{P}(1)$.
	
	Second, by Taylor expansion, $\tau(\theta_0+a_{n}^{-1}t)=\tau(\theta_0)+a_{n}^{-1}D\tau(\theta_0+e_{t}t)t$,
	where $\|e_{t}\|\leq a_{n}^{-1}$. Then $\phi_{1}\{\tau(\theta);n\}$ is equal to
	\begin{eqnarray}
%	 \phi_{1}\{\tau(\theta);n\} &= 
    &\hspace{-1cm}\frac{k_{n}\phi_{2}\{\tau(\theta);n\}}{a_{n}\tau(\theta)}\frac{\int_{t(B_{\delta})}|k_{n}^{-1}D\tau(\theta_0+e_{t}t)t|\ftil_{n}(s_{{\rm obs}}+\veps_{n}\nu\mid\theta_0+a_{n}^{-1}t)\,dt}{\int_{t(B_{\delta})}\ftil_{n}(s_{{\rm obs}}+\veps_{n}\nu\mid\theta_0+a_{n}^{-1}t)\,dt} \leq  \nonumber \\
	&\left( \frac{k_{n}\phi_{2}\{\tau(\theta);n\}}{a_{n}\tau(\theta)}\sup_{\theta\in B_{\delta}}\|k_{n}^{-1}D\tau(\theta)\| \right) \times \\
	&\left(\frac{\int_{t(B_{\delta})}\|t\|a_{n}^{d}\ftil_{n}(s_{{\rm obs}}+\veps_{n}\nu\mid\theta_0+a_{n}^{-1}t)dt}{\int_{t(B_{\delta})}a_{n}^{d}\ftil_{n}(s_{{\rm obs}}+\veps_{n}\nu\mid\theta_0+a_{n}^{-1}t)\,dt}\right),\label{eq3}
	\end{eqnarray}
	where the inequality holds by the triangle inequality. By the expression
	\eqref{eq2} and Lemma 7 in the supplementary material of ~\cite{Li2016},
	the right hand side of \eqref{eq3} is $O_{P}(1)$. Then together
	with $\phi_{2}\{\tau(\theta);n\}=\Theta_{P}(1)$, $\phi_{1}\{\tau(\theta);n\}=o_{P}(1).$
	Therefore the Lemma holds.
	\hfill{$\square$} 
%-------------
 
Define the joint density of $(\theta,s)$ in Algorithm \ref{alg:rejACC} and its approximation, where the s-likelihood is replaced by its Gaussian
limit and $r_{n}(\theta)$ by its truncation, by $q_{\veps}(\theta,s)$
and $\tilde{q}_{\veps}(\theta,s)$. Then
\begin{align*}
q_{\veps}(\theta,s) & =\frac{r_{n}(\theta)f_{n}(s|\theta)K_{\veps_{n}}(s-s_{\rm obs})}{\int_{\mathbb{R}^{p}\times\mathbb{R}^{d}}r_{n}(\theta)f_{n}(s|\theta)K_{\veps_{n}}(s-s_{\rm obs})\,d\theta ds},\\
\tilde{q}_{\veps}(\theta,s) & =\frac{r_{\delta}(\theta)\ftil_{n}(s|\theta)K_{\veps_{n}}(s-s_{\rm obs})}{\int_{\mathbb{R}^{p}\times\mathbb{R}^{d}}r_{\delta}(\theta)\ftil_{n}(s|\theta)K_{\veps_{n}}(s-s_{\rm obs})\,d\theta ds}.
\end{align*}
Let $\tilde{Q}_{\veps}(\theta\in A\mid s_{\rm obs})$ be the approximate confidence distribution function equal to $\int_{A}\int_{\mathbb{R}^{d}}\tilde{q}_{\veps}(\theta,s)\,dsd\theta$. %\ST{is this supposed to be a different function than the one for $\tilde{Q}(\theta \in A \mid s)$ above?} \WL{This depends on $\varepsilon$; the one above doesn't.}
With the transformation $t=t(\theta)$ and $v=v(s)$, let $$\tilde{q}_{\veps,t\nu}(t,v)=\tau_{n}^{-p}r_{\delta}(\theta_0+a_{n,\veps}^{-1}t)\ftil_{n}(s_{\rm obs}+\veps_{n}\nu\mid\theta_0+a_{n,\veps}^{-1}t)K_{\veps}(\nu)$$
be the transformed and unnormalized $\tilde{q}_{\veps}(\theta,s)$, and
$$\tilde{q}_{A,tv}(h)=\int_{A}\int_{\mathbb{R}^{d}}h(t,v)\tilde{q}_{\veps,t\nu}(t,v)\,dvdt$$
for any function $h(\cdot,\cdot)$ in $\mathbb{R}^{p}\times\mathbb{R}^{d}$.
Denote the factor of $\tilde{q}_{\veps,t\nu}(t,v)$, $\tau_{n}^{-p}r_{\delta}(\theta_0+a_{n,\veps}^{-1}t)$,
by $\gamma_{n}(t)$. Let $\gamma=\lim_{n\rightarrow\infty}\tau_{n}^{-p}r_{\delta}(\theta)$
and $\gamma(t)=\lim_{n\rightarrow\infty}\tau_{n}^{-p}r_{\delta}(\theta_0+\tau_{n}^{-1}t)$,
the limits of $\gamma_{n}(t)$ when $a_{n,\veps}=a_{n}$ and
$a_{n,\veps}=\tau_{n}$ respectively. By Condition \ref{initial_upper} and \ref{initial_lower},
$\gamma(t)$ exists and $\gamma$ is non-zero with positive probability.


Next several functions of $t$ and $v$ defined in ~\cite[proofs for Section 3.1]{Li2017}
and relate to the limit of $\tilde{q}_{\veps,t\nu}(t,v)$ are used, including
$g(v;A,B,c)$, $g_{n}(t,v)$, $G_{n}(v)$ and $g_{n}'(t,v)$. 

Furthermore
several functions defined by integration as following are used: for
any $A\in\mathfrak{B}^{p}$, let $$g_{A,r}(h)=\int_{\mathbb{R}^{d}}\int_{t(A)}h(t,v)\gamma_{n}(t)g_{n}(t,v)\,dtdv,$$
$$G_{n,r}(v)=\int_{t(B_{\delta})}\gamma_{n}(t)g_{n}(t,v)\,dt,$$ 
$$q_{A}(h)=\int_{A}\int_{\mathbb{R}^{d}}h(\theta,s)r_{n}(\theta)f_{n}(s\mid\theta)K_{\veps}(s-s_{\rm obs})\veps_{n}^{-d}\,dsd\theta,$$
$$\tilde{q}_{A}(h)=\int_{A}\int_{\mathbb{R}^{d}}h(\theta,s)r_{\delta}(\theta)\ftil_{n}(s\mid\theta)K_{\veps}(s-s_{\rm obs})\veps_{n}^{-d}\,dsd\theta,$$
which generalize those defined in ~\cite[proofs for Section 3.1]{Li2017}
for the case $r_{n}(\theta)=\pi(\theta)$. 


%-------------
\begin{lemma}\label{Alemma2} Assume Condition \ref{par_true}--\ref{kernel_prop}. If $\veps_{n}=o(a_{n}^{-1/2})$, then 
	\begin{enumerate}
		\item[(i)] $\int_{\mathbb{R}^{d}}\int_{t(B_{\delta})}|\tilde{q}_{\veps,t\nu}(t,\nu)-\gamma_{n}(t)g_{n}(t,\nu)|\,dtd\nu=o_{p}(1)$;
		\item[(ii)] $g_{B_{\delta},r}(1)=\Theta_{P}(1);$ 
		\item[(iii)] $\tilde{q}_{B_{\delta},tv}(t^{k_{1}}v^{k_{2}})/\tilde{q}_{B_{\delta},tv}(1)=g_{B_{\delta},r}(t^{k_{1}}v^{k_{2}})/g_{B_{\delta},r}(1)+O_{P}(a_{n,\veps}^{-1})+O_{P}(a_{n}^{2}\veps_{n}^{4})$
		for pairs $(k_{1}, k_{2}) = (0,0), (1,0), (1,1), (0,1)$, and $(0,2)$; 
		\item[(iv)] $\tilde{q}_{B_{\delta}}(1)=$
		\begin{eqnarray*}
		&\hspace{-1cm}\tau_{n}^{p}a_{n,\veps}^{d-p}\left\{ \int_{t(B_{\delta})}\int_{\mathbb{R}^{d}}\gamma_{n}(t)g_{n}(t,\nu)d\tau d\nu+O_{P}(a_{n,\veps}^{-1})+O_{P}(a_{n}^{2}\veps_{n}^{4})\right\} .
		\end{eqnarray*} 
	\end{enumerate}\end{lemma}
\noindent {\it Proof of Lemma \ref{Alemma2}:} 
	These results generalize parts of Lemma A2 in ~\cite{Li2017} (corresponding to items $(i)$ and $(ii)$ above) and Lemma 5 in
	~\cite{Li2016} (corresponding to items $(iii)$ and $(iv)$ above). 
	
	To prove part $(i)$, note that in Lemma A2 of ~\cite{Li2017}  $\gamma_{n}(t)=\pi(\theta_0+a_{n,\veps}^{-1}t)$,
	and $(i)$ holds by expanding $\tilde{q}_{\veps,t\nu}(t,v)$ according to
	the proof of Lemma 5 of ~\cite{Li2016}. 
	For $\gamma_{n}(t)=\tau_{n}^{-p}r_{\delta}(\theta_0+a_{n,\veps}^{-1}t)$ this can be similarly proved by changing the terms involving $\pi(\theta)$ in equations (10) and (11) in the supplements of ~\cite{Li2016}. Equation (10)
	is replaced by 
	\[
	\frac{\gamma_{n}(t)}{\mid A(\theta+a_{n,\veps}^{-1}t)\mid^{1/2}}=\frac{\gamma_{n}(t)}{\mid A(\theta)\mid^{1/2}}+a_{n,\veps}^{-1}\gamma_{n}(t)D\frac{1}{\mid A(\theta+e_{t})\mid^{1/2}}t,
	\]
	where $\|e_{\tau}\|\leq\delta$, and this leads to replacing $\pi(\theta_0)\int_{\tau(B_{\delta})\times\mathbb{R}^{d}}g_{n}(t,\nu)dtd\nu$
	in equation (11) by $\int_{\tau(B_{\delta})\times\mathbb{R}^{d}}\gamma_{n}(t)g_{n}(t,\nu)dtd\nu$.
	These changes have no effect on the arguments therein since $\sup_{t\in t(B_{\delta})}\gamma_{n}(t)=O_{P}(1)$
	by Condition \ref{initial_upper}. Therefore $(i)$ holds.
	
	For (ii), By Condition \ref{initial_lower} and Lemma A2 of ~\cite{Li2017}, there exists a $\delta'<\delta$
	such that $\inf_{t\in t(B_{\delta'})}\gamma_{n}(t)=\Theta_{p}(1)$
	and $\int_{\mathbb{R}^{d}}\int_{t(B_{\delta'})}g_{n}(t,\nu)\,dtdv=\Theta_{p}(1)$.
	Then since $g_{B_{\delta},r}(1)\geq\inf_{t\in t(B_{\delta'})}\gamma_{n}(t)\int_{\mathbb{R}^{d}}\int_{t(B_{\delta'})}g_{n}(t,\nu)\,dtd\nu$,
	(ii) holds.
	
	For $(iii)$, if $(k_1,k_2)=(1,0)$ then $\tilde{q}_{B_{\delta},tv}(t)/\tilde{q}_{B_{\delta},tv}(1)$
	can be expanded by following the arguments in the proof of Lemma 5
	of ~\cite{Li2016}. For the other pairs of $(k_1,k_2)$, $\tilde{q}_{B_{\delta},tv}(t^{k_{1}}v^{k_{2}})/\tilde{q}_{B_{\delta},tv}(1)$,
	can be expanded similarly as in the proof of Lemma 4 from ~\cite{Li2017}.
	
	For $(iv)$, $\gamma_{n}(t)$ plays the same role as $\pi(\theta)$
	in the proof of Lemma 5 in ~\cite{Li2016}, and the arguments therein
	can be followed exactly. The term $\tau_{n}^{p}$ is from the definition
	of $\gamma_{n}(t)$ that $r_{n}(\theta_0+a_{n,\veps}^{-1}t)=\tau_{n}^{p}\gamma_{n}(t)$.
	\hfill{$\square$} 
	
	
%-------------

Recall the definition of the estimator $\theta_{\veps} = \int \theta dQ_{\veps}(\theta \mid s_{obs}) d\theta$. Define the expectation of $\theta$ with distribution $\tilde{Q}_{\veps}(\theta\in A\mid s_{\rm obs})$
as $\tilde{\theta}_{\veps}$ and the expectation of the regression adjusted values, $\theta^*$ 
with density $\tilde{q}_{\veps}(\theta,s)$ as $\tilde{\theta}_{\veps}^{*}$.
Let $E_{G,r}(\cdot)$ be the expectation with the density $G_{n}(v)^{({\rm norm})}$,
and $E_{G,r}\{h(v)\}$ can be written as $g_{B_{\delta},r}\{h(v)\}/g_{B_{\delta},r}(1)$.
Let $\psi(\nu)=k_{n}^{-1}\beta_{0}\{A(\theta_{0})^{1/2}W_{\rm obs}+a_{n}\veps_{n}\nu\}$,
where $k_{n}=1$, if $c_{\veps}<\infty$, and $a_{n}\veps_{n}$,
if $c_{\veps}=\infty$. 

%-------------

\begin{lemma}\label{Alemma3} Assume Condition \ref{par_true}--\ref{initial_gradient} and \ref{kernel_prop}. Then if $\veps_{n}=o(a_{n}^{-1/2})$, 
	\begin{enumerate}
		\item[(i)] $\tilde{\theta}_{\veps}=\theta_{0}+a_{n}^{-1}\beta_{0}A(\theta_{0})^{1/2}W_{\rm obs}+\veps_{n}\beta_{0}E_{G_{n},r}(\nu)+r_{1}$,
		where $r_{1}=o_{P}(a_{n}^{-1})$; 
		\item[(ii)] $\tilde{\theta}_{\veps}^{*}=\theta_{0}+a_{n}^{-1}\beta_{0}A(\theta_{0})^{1/2}w_{\rm obs}+\veps_{n}(\beta_{0}-\beta_{\veps})E_{G_{n},r}(\nu)+r_{2}$,
		where $r_{2}=o_{P}(a_{n}^{-1})$. 
	\end{enumerate}\end{lemma}

%\ST{Idea of this lemma:} Expand $g_{B_{\delta,r}}(t)$ and then plug this in to the equations for $\tilde{\theta}_{\veps}$ and $\tilde{\theta}_{\veps}^*$ which are dependent upon Lemma 1. \ST{[Question: Lemma 1 requires Condition 1 holds. Where/when is this established? ]} \WL{The proof below doesn't require Lemma 1; Lemma 1 requires Condition 3--8, not Condition 1}

\noindent {\it Proof of Lemma \ref{Alemma3}:} 
	These results generalize Lemma A3(c) and Lemma A5(c) in ~\cite{Li2017} in the sense of permitting use of a data-dependent $r_n(\theta)$, however here we are only considering $\veps_n = o(a_n^{-1/2})$ in contrast to Lemma A5(c) in ~\cite{Li2017} which assumes the less stringent condition that $\veps_n = o(a_n^{-3/5})$.
	
	With the transformation $t=t(\theta)$, by Lemma 2, if $\veps_{n}=o(a_{n}^{-1/2})$,
	\begin{eqnarray}
	\begin{cases}
	\tilde{\theta}_{\veps}=\theta_{0}+a_{n,\veps}^{-1}\tilde{q}_{B_{\delta},t\nu}(t)/\tilde{q}_{B_{\delta},t\nu}(1)=\theta_{0}+a_{n,\veps}^{-1}g_{B_{\delta},r}(t)/g_{B_{\delta},r}(1)+o_{p}(a_{n}^{-1}),\\
	\tilde{\theta}_{\veps}^{*}=\theta_{0}+a_{n,\veps}^{-1}\tilde{q}_{B_{\delta},t\nu}(t)/\tilde{q}_{B_{\delta},t\nu}(1)-\veps_{n}\beta_{\veps}\tilde{q}_{B_{\delta},t\nu}(\nu)/\tilde{q}_{B_{\delta},t\nu}(1)\\
	\hspace{5mm} =\theta_{0}+a_{n,\veps}^{-1}g_{B_{\delta},r}(t)/g_{B_{\delta},r}(1)-\veps_{n}\beta_{\veps}E_{a_{n},r}(\nu)+o_{p}(a_{n}^{-1}),
	\end{cases}\label{eq4}
	\end{eqnarray}
	where the remainder term comes from the fact that $(a_{n,\veps}^{-1}+\veps_{n})\left\{ O_{p}(a_{n,\veps}^{-1})+O_{p}(a_{n}^{2}\veps_{n}^{4})\right\} =o_{p}(a_{n}^{-1})$.
	
	First the leading term of $g_{B_{\delta},r}(t\nu^{k})$ is derived
	for $k=0$ or $1$. The case of $k=1$ will be used later. Let $t'=t-\psi(\nu)$,
	then 
	\begin{align*}
	g_{B_{\delta},r}(t\nu^{k_{2}}) & =\int_{\mathbb{R}^{d}}\int_{t(B_{\delta})}\{t'+\psi(\nu)\}\nu^{k_{2}}\gamma_{n}(t)g_{n}(t,\nu)\,dtd\nu\\
	& =\int_{\mathbb{R}^{d}}\psi(\nu)\nu^{k_{2}}G_{n,r}(\nu)\,d\nu+\int_{\mathbb{R}^{d}}\int_{t(B_{\delta})}t'\nu^{k_{2}}\gamma_{n}(t)g_{n}(t,\nu)\,dtd\nu.
	\end{align*}
	By matrix algebra, it is straightforward to show that $g_{n}(t,v)=N\{t;\psi(v),k_{n}^{-2}I(\theta_{0})^{-1}\}G_{n}(v)$.
	Then with the transformation $t'$, we have
	\begin{align*}
	& g_{B_{\delta},r}(t\nu^{k_{2}})-\int_{\mathbb{R}^{d}}\psi(\nu)\nu^{k_{2}}G_{n,r}(\nu)\,d\nu\\
	= & \int_{\mathbb{R}^{d}}\int_{t(B_{\delta})-\psi(\nu)}t'\nu^{k_{2}}\gamma_{n}\{\psi(\nu)+t'\}N\left\{ t';0,k_{n}^{-2}I(\theta_{0})^{-1}\right\} G_{n}(\nu)\,dt'd\nu.
	\end{align*}
	By applying the Taylor expansion on $\gamma_{n}\{\psi(\nu)+t'\}$,
	the right hand side of the above equation is equal to 
	\begin{eqnarray}
	&  & \int_{\mathbb{R}^{d}}\int_{t(B_{\delta})-\psi(\nu)}t'N\{t';0,k_{n}^{-2}I(\theta_{0})^{-1}\}\,dt'\cdot\gamma_{n}\{\psi(\nu)\}\nu^{k_{2}}G_{n}(\nu)\,d\nu\nonumber \\
	&  & +\int_{\mathbb{R}^{d}}\int_{t(B_{\delta})-\psi(\nu)}t'{}^{2}D_{t}\gamma_{n}\{\psi(\nu)+e_{t}\}N\{t';0,k_{n}^{-2}I(\theta_{0})^{-1}\}\,dt'\cdot\nu^{k_{2}}G_{n}(\nu)d\nu\nonumber \\
	& = & k_{n}^{-1}\int_{\mathbb{R}^{d}}\int_{Q_{v}}t''N\{t'';0,I(\theta_{0})^{-1}\}\,dt''\cdot\gamma_{n}\{\psi(\nu)\}\nu^{k_{2}}G_{n}(\nu)\,d\nu\nonumber \\
	&  & +k_{n}^{-2}\int_{\mathbb{R}^{d}}\int_{Q_{v}}t''^{2}D_{t}\gamma_{n}\{\psi(\nu)+e_{t}\}N\{t'';0,I(\theta_{0})^{-1}\}\,dt''\cdot\nu^{k_{2}}G_{n}(\nu)\,d\nu,\label{eq5}
	\end{eqnarray}
	where $Q_{v}=\left\{ a_{n}(\theta-\theta_{0})-k_{n}\psi(\nu)\mid\theta\in B_{\delta}\right\} $
	and $t''=k_{n}t'$. 
	%%%%TO DO: clarify ambiguity in next statement, then move on to Lemma 4 and 5 and proof of Thm 1. 
	Since $Q_{v}$ can be written as $\left\{ a_{n}(\theta-\theta_{0}-\beta_{0}\veps_{n}\nu)-\beta_{0}A(\theta_{0})^{1/2}W_{\rm obs}\mid\theta\in B_{\delta}\right\} $,
	it converges to $\mathbb{R}^{p}$ for any fixed $v$ with probability one using the dominated convergence theorem. %\ST{[I don't think this convergence statement makes mathematical sense. Should explain how to change limits of integration when substituting $t''$]}
	Then $\int_{Q_{v}}t''N\{t'';0,\tau(\theta_{0})^{-1}\}\,dt''=o_{P}(1)$
	for fixed $v$, and by the continuous mapping theorem and Condition \ref{initial_upper}, the
	first term in the right hand side of \eqref{eq5} is of the order
	$o_{p}(k_{n}^{-1})$. The second term is bounded by 
	\[
	k_{n}^{-2}\sup_{t\in\mathbb{R}}\|D_{t}\gamma_{n}(t)\|\int_{\mathbb{R}^{p}}\|t''\|^{2}N\{t'';0,I(\theta_{0}^{-1})\}\,dt''\int_{\mathbb{R}^{d}}\nu^{k_{2}}G_{n}(\nu)\,d\nu,
	\]
	which is of the order $O_{p}(k^{-2}\tau_{n}/a_{n,\veps})$ by Condition \ref{initial_gradient}.
	Therefore 
	\begin{align}
	g_{B_{\delta},r}(t\nu^{k_{2}}) & =\int_{\mathbb{R}^{d}}\psi(\nu)\nu^{k_{2}}G_{n}(\nu)d\nu+o_{P}(k_{n}^{-1}).\label{eq6}
	\end{align}
	By algebra, $k_{n}=a_{n,\veps}^{-1}a_{n}$, and 
	\begin{eqnarray}
	&  & \int_{\mathbb{R}^{d}}\psi(\nu)\nu^{k_{2}}G_{n}(\nu)d\nu\nonumber \\
	& = & a_{n,\veps}\beta_{0}\{a_{n}^{-1}A(\theta_{0})^{1/2}W_{\rm obs}\int_{\mathbb{R}^{d}}\nu^{k_{2}}G_{n,r}(\nu)\,d\nu+\veps_{n}\int_{\mathbb{R}^{d}}\nu^{k_{2}+1}G_{n,r}(\nu)\,d\nu\}.\label{eq7}
	\end{eqnarray}
	Then ($i$) and ($ii$) in the Lemma holds by plugging the expansion
	of $g_{B_{\delta},r}(t)$ into \eqref{eq4}.
	
	\hfill{$\square$} 
	
%-------------	

\begin{lemma}\label{Alemma3.5} Assume Condition \ref{par_true}, \ref{initial_upper}, \ref{sum_conv}--\ref{sum_approx_tail}. Then as $n\rightarrow\infty$, 
	\begin{enumerate}
		\item[(i)] For any $\delta<\delta_{0}$, $r_{B_{\delta}^{c}}(1)$ and $\tilde{q}_{B_{\delta}^{c}}(1)$
		are $o_{p}(\tau_{n}^{p})$. More specifically, they are of the order
		$O_{p}\left(\tau_{n}^{p}e^{-a_{n,\veps}^{\alpha_{\delta}}c_{\delta}}\right)$
		for some positive constants $c_{\delta}$ and $\alpha_{\delta}$ depending
		on $\delta$.
		\item[(ii)] $q_{B_{\delta}}(1)=\tilde{q}_{B_{\delta}}(1)\{1+O_{p}(\alpha_{n}^{-1})\}$
		and $\sup_{A\subset B_{\delta}}|q_{A}(1)-\tilde{q}_{A}(1)|/\tilde{q}_{B_{\delta}}(1)=O_{p}(\alpha_{n}^{-1})$; 
		\item[(iii)] if $\veps_{n}=o(a_{n}^{-1/2})$, then $\tilde{q}_{B_{\delta}}(1)$ and
		$r_{B_{\delta}}(1)$ are $\Theta_{P}(\tau_{n}^{p}a_{n,\veps}^{d-p})$,
		and thus $\tilde{q}_{\mathcal{P}_{0}}(1)$ and $q_{\mathcal{P}_{0}}(1)$
		are $\Theta_{P}(\tau_{n}^{p}a_{n,\veps}^{d-p})$; 
		\item[(iv)] if $\veps_{n}=o(a_{n}^{-1/2})$, $\theta_{\veps}=\tilde{\theta}_{\veps}+o_{p}(a_{n,\veps}^{-1}).$
		If $\veps_{n}=o(a_{n}^{-3/5}),$ $\theta_{\veps}=\tilde{\theta}_{\veps}+o_{P}(a_{n}^{-1}).$
	\end{enumerate} \end{lemma}
\noindent {\it Proof of Lemma \ref{Alemma3.5}:} 
	This generalizes Lemma 2 in the supplements of ~\cite{Li2017}. The arguments therein
	can be followed exactly, by Condition \ref{initial_upper} and the fact that regarding $\pi(\theta)$,
	only the condition $\sup_{\theta\in\mathbb{R}^{p}}\pi(\theta)<\infty$
	is used.
	
	\hfill{$\square$} 
	
%-------------	
\begin{lemma}\label{Alemma4} 
Assume Condition \ref{par_true}, \ref{initial_upper}, \ref{sum_conv}--\ref{sum_approx_tail}. 
	\begin{enumerate}
		\item[(i)] For any $\delta<\delta_{0}$, $Q_{\veps}(\theta\in B_{\delta}^{c}\mid s_{\rm obs})$
		and $\tilde{Q}_{\veps}(\theta\in B_{\delta}^{c}\mid s_{\rm obs})$ are $o_{p}(1)$; 
		\item[(ii)] There exists some $\delta<\delta_{0}$ such that 
		\[
		\sup_{A\in\mathfrak{B}^{p}}|Q_{\veps}(\theta\in A\cap B_{\delta}\mid s_{\rm obs})-\tilde{Q}_{\veps}(\theta\in A\cap B_{\delta}\mid s_{\rm obs})|=o_{p}(1);
		\]
		\item[(iii)] $a_{n,\veps}(\theta_{\veps}-\tilde{\theta}_{\veps})=o_{p}(1)$ . 
	\end{enumerate}\end{lemma}
\noindent {\it Proof of Lemma \ref{Alemma4}:} 
	This lemma generalizes Lemma A3 of~\cite{Li2017}. The proof of Lemma A3 of~\cite{Li2017} only needs Lemma 3 and 5 from~\cite{Li2016} to hold. The result that  $q_{B_{\delta}^{c}}\{h(\theta)\}=O_{p}(\tau_{n}^{p}e^{-a_{n,\veps}^{\alpha_{\delta}}c_{\delta}})$
	for some positive constants $\alpha_{\delta}$ and $c_{\delta}$, which generalizes the case of $r_{n}(\theta)=\pi(\theta)$ in Lemma 3 of~\cite{Li2016}, holds by Condition \ref{initial_upper}, since the latter only uses the fact that 
	%$\pi(\theta)$ it only uses the fact that 
	$\sup_{\theta\in B_{\delta}^{c}}\pi(\theta)<\infty$.
	Then the arguments in the proof of Lemma 3 in~\cite{Li2016} can be followed exactly, despite the term $\tau_{n}^{p}$ that is not included in the order of $\pi_{B_{\delta}^{c}}\{h(\theta)\}$, since $Q_{\veps}(\theta\in A\mid s_{{\rm obs}})$
	is the ratio $q_{A}(1)/q_{\mathbb{R}^{p}}(1)$. Since Lemma 5 in~\cite{Li2016} has been generalized by Lemma \eqref{Alemma2} above, the arguments of the proof of Lemma A3 in~\cite{Li2017} can be followed exactly.
	\hfill{$\square$} \\
	

%This result generalizes %the case ($i$) of 
%Proposition 1 in~\cite{Li2017} in the sense of permitting use of a data-dependent $r_n(\theta)$, however for Theorem 1 we need only consider $\veps_n = o(a_n^{-1/2})$.
With the above lemmas holding for $\veps_n = o(a_n^{-1/2})$, lines for
proving %case $(i)$ of 
Proposition 1 in~\cite{Li2017} can be followed  
exactly to finish the proof of Theorem \ref{thm:ACC_limit_small_bandwidth}. 

%-----------------------------------------------------------------------------------------------------------

\section{Proof of Theorem 3}

\begin{lemma}\label{Alemma5} 
Assume Condition \ref{par_true}--\ref{cond:likelihood_moments}. If $\veps_{n}=o_{p}(a_{n}^{-3/5})$, then $a_{n}\veps_{n}(\beta_{\veps}-\beta_{0})=o(1)$.
\end{lemma}
\noindent {\it Proof of Lemma \ref{Alemma5}:} 
	This generalizes Lemma A4 in ~\cite{Li2017} by replacing $\pi(\theta_{0}+a_{n,\veps}^{-1}t)$
	therein with $\gamma_{n}(t)$. By Condition \ref{initial_upper} and the arguments in the proof
	of Lemma A4 in ~\cite{Li2017}, it can be shown that 
	\[
	\frac{q_{\mathbb{R}^{p}}\{(\theta-\theta_{0})^{k_{1}}(s-s_{\rm obs})^{k_{2}}\}}{q_{\mathbb{R}^{p}}(1)}=a_{n,\veps}^{-k_{1}}\veps_{n}^{-k_{2}}\left\{ \frac{\tilde{q}_{B_{\delta},tv}(t^{k_{1}}\nu^{k_{2}})}{\tilde{q}_{B_{\delta},tv}(1)}+O_{p}(\alpha_{n}^{-1})\right\} .
	\]
	Then by Lemma 2 $(iii)$, the right hand side of the above is equal
	to 
	\[
	a_{n,\veps}^{-k_{1}}\veps_{n}^{-k_{2}}\left\{ \frac{g_{B_{\delta},r}(t^{k_{1}}\nu^{k_{2}})}{g_{B_{\delta},r}(1)}+O_{p}(a_{n,\veps}^{-1})+O_{p}(a_{n}^{2}\veps_{n}^{4})+O_{p}(\alpha_{n}^{-1})\right\} .
	\]
	Since $\beta_{\veps}=\text{Cov}_{\veps}(\theta,S_{n})\text{Var}_{\veps}(S_{n})^{-1}$,
	\begin{align*}
	a_{n}\veps_{n}(\beta_{\veps}-\beta_{0})= & k_{n}\left[\frac{g_{B_{\delta},r}(t\nu)}{g_{B_{\delta},r}(1)}-\frac{g_{B_{\delta},r}(t)g_{B_{\delta},r}(\nu)}{g_{B_{\delta},r}(1)^{2}}+o_{p}(k_{n}^{-1})\right]\times\\
	& \qquad\left[\frac{g_{B_{\delta},r}(\nu\nu^{T})}{g_{B_{\delta},r}(1)}-\frac{g_{B_{\delta},r}(\nu)g_{B_{\delta},r}(\nu)^{T}}{g_{B_{\delta},r}(1)^{2}}+o_{p}(k_{n}^{-1})\right]^{-1}-a_{n}\veps_{n}\beta_{0},
	\end{align*}
	where the equations that $a_{n,\veps}^{-1}k_{n}=o(1)$, $a_{n}^{2}\veps_{n}^{4}k_{n}=o(p)$,
	and $\alpha_{n}^{-1}k_{n}=o(a_{n}^{-2/5}k_{n})=o(1)$ are used. By
	algebra, the right hand side of the equation above can be rewritten
	as 
	\begin{eqnarray*}
		&  & \left\{ \frac{g_{B_{\delta},r}\{(k_{n}t-a_{n}\veps_{n}\beta_{0}\nu)\nu\}}{g_{B_{\delta},r}(1)}-\frac{g_{B_{\delta},r}(k_{n}t-a_{n}\veps_{n}\beta_{0}\nu)g_{B_{\delta},r}(\nu)}{g_{B_{\delta},r}(1)^{2}}+o_{p}(1)\right\} \times\\
		&  & \qquad\left\{ E_{G,r}(\nu\nu^{T})-E_{G,r}(\nu)E_{G,r}(\nu)^{T}+o_{p}(k_{n}^{-1})\right\} ^{-1}.
	\end{eqnarray*}
	By plugging \eqref{eq6} and \eqref{eq7} in the above, $a_{n}\veps_{n}(\beta_{\veps}-\beta_{0})$
	is equal to 
	\begin{eqnarray*}
		&  & \left\{ E_{G,r}(\nu)\beta_{0}A(\theta_{0})^{1/2}W_{\rm obs}-E_{G,r}(\nu)\beta_{0}A(\theta_{0})^{1/2}W_{\rm obs}+o_{p}(1)\right\} \times \\
		&  & \hspace{1cm}
		\{\text{Var}_{G,r}(\nu)+o_{p}(k_{n}^{-1})\}^{-1}\\
		&=& o_{P}(1)\{\text{Var}_{G,r}(\nu)+o_{p}(k_{n}^{-1})\}^{-1}.
	\end{eqnarray*}
	Since
	\begin{align*}
	\text{Var}_{G,r}(\nu) & \geq\frac{\inf_{t\in t(B_{\delta'})}\gamma_{n}(t)}{g_{B_{\delta},r}(1)}\int_{\mathbb{R}^{d}}\int_{t(B_{\delta'})}\{\nu-E_{G,r}(\nu)\}^{2}g_{n}(t,\nu)\,dtd\nu,
	\end{align*}
	where $\delta'$ is defined in the proof of Lemma \ref{Alemma2}(ii),
	we have $\text{Var}_{G,r}(\nu)^{-1}=\Theta_{p}(1)$. Therefore $a_{n}\veps_{n}(\beta_{\veps}-\beta_{0})=o_{p}(1)$.
	\hfill{$\square$} 
%-------------

% \pagebreak 
\begin{lemma}\label{Alemma6} Results generalizing Lemma A5 in ~\cite{Li2017},
	$i.e.$ replacing $\Pi_{\veps}$ and $\tilde{\Pi}_{\veps}$ therein with
	$Q_{\veps}$ and $\tilde{Q}_{\veps}$, hold. \end{lemma} 
\noindent {\it Proof of Lemma \ref{Alemma6}:} 
	In ~\cite{Li2017}, the proof of Lemma A5 in ~\cite{Li2017} requires Lemma A4 and Lemma 2 in the supplements therein to hold. Since we have show that their generalized results hold for $\veps_{n}=o_{p}(a_{n}^{-1/2})$, (see Lemma \ref{Alemma5} and Lemma \ref{Alemma3.5} above),
    the proof of this lemma for $\veps_{n}=o_{p}(a_{n}^{-3/5})$ follows the same arguments in~\cite{Li2017}, replacing $\pi(\theta)$ with a $r_n(\theta)$ that satisfies conditions \ref{par_true} -- \ref{initial_gradient}.

 	\hfill{$\square$} 
%-------------

% \begin{lemma}\label{Alemma7} Results generalizing Lemma 10 in ~\cite{Li2017}
% 	hold. \end{lemma}
% \ST{[This references the original 2015 arXiv version titled "Behavior of ABC for Big Data" which is the only doc containing a lemma 10. I'm unsure as to the role I though this lemma played in supporting Theorem 3...]}

% \noindent {\it Proof of Lemma \ref{Alemma7}:} 
% 	The same arguments can be followed.
	
% 	\hfill{$\square$} 
% %-------------

With all above lemmas, the proof of Theorem \ref{thm:ACC_limit_large_bandwidth} holds by following the same arguments as those in the proof of Theorem 1 in ~\cite{Li2017}.



%\bibliographystyle{nessart-number}
%\bibliography{ACC_paper_2021}	

%\end{document}